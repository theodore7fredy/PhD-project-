%%%%%%%%%%%%%%%%%%%%%%%%%%%%%%%%%%%%%%%%%%%%%%%%%%%%%%%%%%%%%%%%%%%%%%%%%%%%%%%
% PROJET DE THESE EN DYNAMIQUE QUANTIQUE - adHOPS/DadHOPS
% Rédigé Nana Engo pour Théodore Fredy Goumai
% Date : 20 Septembre 2025
% Version révisée avec intégration agrivoltaïsme et toxicité
%%%%%%%%%%%%%%%%%%%%%%%%%%%%%%%%%%%%%%%%%%%%%%%%%%%%%%%%%%%%%%%%%%%%%%%%%%%%%%%

\documentclass[12pt, a4paper]{article}

\usepackage[utf8]{inputenc}      
\usepackage[T1]{fontenc}         
\usepackage[french]{babel}       
\usepackage[margin=2.cm]{geometry} 
%
\usepackage{amsmath, amssymb, physics}
%
\usepackage{graphicx,booktabs,tabularx,hyperref,xcolor,tablefootnote} 
%
\hypersetup{
    colorlinks=true,             % Liens colorés plutôt qu'encadrés
    linkcolor=blue!50!black,     % Couleur des liens internes (sections, figures)
    citecolor=green!50!black,    % Couleur des citations (non utilisé ici, mais bonne pratique)
    urlcolor=blue!80!black,      % Couleur des URLs
    pdftitle={Projet de Thèse - Dynamique Quantique Non-Markovienne},
    pdfauthor={Théodore Fredy Goumai}
}
%
% Ramener la typologie française à la typologie standard (anglo-saxone)
\frenchbsetup{StandardLayout=true}

%
\title{\huge Dynamique quantique non-markovienne pour une photonique durable : conception de matériaux bio-inspirés performants et biocompatibles pour l'agrivoltaïsme}
\author{
    Théodore Fredy Goumai (Doctorant) \\
    \\
    \textit{Sous la direction de} \\
    J.-P. Tchapet Njafa, PhD \\
    S. G. Nana Engo, Professeur
}
\date{Octobre 2025}


%%%%%%%%%%%%%%%%%%%%%%%%%%%%%%%%%%%%%%%%%%%%%%%%%%%%%%%%%%%%%%%%%%%%%%%%%%%%%%%
% DÉBUT DU DOCUMENT
%%%%%%%%%%%%%%%%%%%%%%%%%%%%%%%%%%%%%%%%%%%%%%%%%%%%%%%%%%%%%%%%%%%%%%%%%%%%%%%
\begin{document}

\maketitle
\thispagestyle{empty} % Pas de numéro sur la page de titre
\newpage

\tableofcontents % Ajout d'une table des matières
\newpage
\setcounter{page}{1} % Réinitialisation du compteur de pages


\section{Contexte et justification scientifique}

Le développement de matériaux performants et durables pour la conversion photovoltaïque constitue un défi majeur pour la transition énergétique. Au cœur de ces technologies, les processus fondamentaux de transport et de séparation de charge reposent sur une dynamique quantique complexe, intimement couplée à un environnement thermo-vibrationnel \cite{ye2012, mohs2008}. Cet environnement, loin d'être un simple bain dissipatif, impose souvent des effets de mémoire, dits non-markoviens, que les approches théoriques traditionnelles peinent à capturer fidèlement. Les approximations markoviennes (e.g. équations de Redfield), bien que utiles, s'avèrent souvent insuffisantes pour décrire l'influence des couplages forts et des corrélations temporelles, occultant ainsi le rôle critique des cohérences quantiques dans l'efficacité des dispositifs \cite{tao2020, dijkstra2010, Worster2019}.

Parallèlement, l'émergence de l'agrivoltaïsme - combinaison synergique de production agricole et énergétique - pose de nouveaux défis scientifiques : comment optimiser simultanément le rendement photovoltaïque et la productivité végétale sous les panneaux ? Cette question implique une compréhension fine des mécanismes de transfert d'énergie quantique dans les deux systèmes, ainsi que des considérations sur la durabilité et la toxicité des matériaux utilisés, en évitant notamment les composés à base de métaux lourds comme le CdTe \cite{archet2018, leb2016}.

Pour surmonter ces limitations, la méthode de la \textbf{hiérarchie adaptative d'états purs (adHOPS)} a émergé comme une solution numériquement exacte et particulièrement puissante \cite{Suess2014}. Son avantage déterminant réside dans sa capacité à exploiter la localisation dynamique des excitations quantiques et à opérer une réduction adaptative de la base de calcul. Cette approche lui confère une scalabilité quasi-indépendante de la taille du système ($\mathcal{O}(1)$) pour de grands agrégats \cite{varvelo2021, Citty2024}, un atout majeur qui rend possible l'étude de la dynamique quantique ouverte à l'échelle mésoscopique (centaines de chromophores) avec un coût computationnel maîtrisé \cite{zheng2021}. Les développements récents, notamment la méthode \textbf{DadHOPS (Dyadic adHOPS)} pour le calcul d'observables spectroscopiques \cite{Gera2023, Chen2022a} et l'implémentation de ces approches dans des frameworks open-source comme \texttt{MesoHOPS} \cite{Citty2024}, démocratisent leur accès et permettent d'envisager des recherches de pointe même dans des environnements aux ressources de calcul modestes \cite{johansson2012}.

\section{Problématique de recherche}

Comment modéliser avec précision et exploiter les interactions fortes et les effets de mémoire non-markoviens dans la dynamique quantique du transport et de la séparation de charges pour concevoir des matériaux photovoltaïques à la fois performants, durables et adaptés aux applications agrivoltaïques ? Plus spécifiquement :
\begin{itemize}
\item Comment l'inclusion explicite de la multiplicité d'états électroniques, notamment les états de transfert de charge (CT) de haute énergie, peut-elle nous guider vers la conception rationnelle de matériaux photovoltaïques et bio-inspirés plus performants ?
\item Comment les concepts transfert d'énergie d'excitation (EET) et de cohérence quantique peuvent-ils éclairer l'impact spectral  de la filtration de la lumière par les panneaux sur la photosynthèse des cultures en dessous ?
\item Comment les principes de l'ingénierie quantique des matériaux (design moléculaire, géométrie des agrégats) peuvent-ils inspirer le développement de matériaux OPV non toxiques, biodégradables ou biocompatibles ?
\end{itemize}

\section{Objectifs de la thèse}

Ce projet s'articule autour de trois axes principaux, visant à la fois une compréhension fondamentale et des retombées applicatives concrètes.

% \subsection{Analyse de l'impact des états de haute énergie dans la séparation de charge}
%
% L'objectif est d'atteindre un niveau de réalisme sans précédent dans la modélisation des systèmes quantiques ouverts complexes.
% \begin{itemize}
%     \item \textbf{Étude de dyades donneur-accepteur innovantes.} Nous utiliserons adHOPS pour quantifier l'influence des états CT de haute énergie sur la cinétique et l'efficacité de la séparation de charge \cite{lee2015, zhang2022_NFA}.
%
%     \item \textbf{Ciblage de matériaux à haut potentiel.} Les simulations se concentreront sur des matériaux bio-inspirés (dérivés de curcuminoïdes, petites molécules fluorées) dont le potentiel de rendement photovoltaïque a été estimé à plus de 20\% \cite{archet2018, firdaus2019, leb2016}.
%
%     \item \textbf{Intégration des critères de durabilité.} Nous développerons des descripteurs quantiques de stabilité et d'innocuité (degré de séparation de charge, énergie de recombination) pour guider la sélection de matériaux non toxiques \cite{Zeng2025, Shi2025}.
% \end{itemize}
%
% \subsection{Optimisation du transfert d'énergie d'excitation dans des systèmes biomimétiques}
%
% Ici, l'objectif est de s'inspirer de l'efficacité remarquable des systèmes naturels.
% \begin{itemize}
%     \item \textbf{Simulation de complexes photosynthétiques.} Nous modéliserons l'efficacité et la robustesse du transfert d'énergie dans des complexes de référence comme FMO et LH2 \cite{wang2018, Harush2023, Pal2025}, en faisant varier les densités spectrales et les géométries avec adHOPS pour reproduire des conditions réalistes \cite{Zhang2021}.
%
%     \item \textbf{Au-delà des modèles standards.} Nous irons au-delà des densités spectrales de type Drude-Lorentz pour intégrer des fonctions de corrélation plus complexes \cite{rangel2002}, issues de calculs de dynamique moléculaire, afin de mieux représenter la structuration de l'environnement protéique \cite{lambert2023}.
%
%     \item \textbf{Vers des règles de conception.} L'objectif ultime est d'extraire des principes de conception clairs et exploitables pour guider la synthèse de nouveaux matériaux biomimétiques optimisés pour le transport d'énergie \cite{Gauger2024, Peter2024}.
% \end{itemize}

\subsection{Renforcer le cadre théorique et méthodologique}

L'objectif est d'atteindre un niveau de réalisme sans précédent dans la modélisation des systèmes quantiques ouverts complexes.
\begin{itemize}
    \item \textbf{Modélisation de l'environnement au-delà de Drude-Lorentz.} Plutôt que de postuler des densités spectrales analytiques, nous calculerons les fonctions de corrélation du bain directement à partir de la \textbf{dynamique moléculaire \textit{ab initio} (AIMD)} des chromophores dans leur environnement réel \cite{rangel2002, lee2015}. De plus, nous inclurons \textbf{explicitement les résidus d'acides aminés clés} dans la partie quantique des calculs (QM/MM), car il a été démontré qu'ils modifient significativement les énergies d'excitation et le caractère des états, notamment par un décalage vers le rouge (\textit{redshift}) \cite{wang2018,Volpert2023}.

    \item \textbf{Pertinence des méthodes numériques (adHOPS/MesoHOPS).} Le choix de la méthode \textbf{adHOPS} est justifié par sa scalabilité quasi-indépendante de la taille du système ($\mathcal{O}(1)$) pour de grands agrégats, rendue possible par l'exploitation de la localisation dynamique des excitons \cite{varvelo2021, Citty2024}. L'utilisation de la bibliothèque open-source \texttt{MesoHOPS} \cite{Citty2024} permettra d'étudier des systèmes mésoscopiques (centaines de chromophores) inaccessibles aux méthodes HEOM standards. De plus, l'extension \textbf{DadHOPS} sera utilisée pour calculer directement des observables spectroscopiques complexes (e.g., 2DES) afin de confronter nos simulations aux données expérimentales \cite{Gera2023, Chen2022a}.
\end{itemize}

\subsection{Conception quantique de systèmes agrivoltaïques optimaux}

Cet axe vise à modéliser pour la première fois la culture sous panneau comme un système quantique ouvert.
\begin{itemize}
    \item \textbf{Modélisation du système quantique ouvert "culture-panneau".} Nous formaliserons le panneau OPV comme un \textbf{"bain de photons filtré"}. La densité spectrale solaire $J_{soleil}(\omega)$ est modifiée par la fonction de transmission du panneau $T(\omega)$, créant une source de pompage pour les plantes $J_{plante}(\omega) = T(\omega) \cdot J_{soleil}(\omega)$ \cite{Gong2024, Shi2025a}.

    \item \textbf{Lien avec la productivité végétale via l'ETR.} Nous utiliserons le modèle mécaniste du \textbf{Taux de Transfert d'Électrons (ETR) ou Taux de Transfert d'Électrons photosynthétique} de \cite{ye2012}, qui dépend explicitement de la "section efficace d'absorption" $\sigma_{ik}(\omega)$ des pigments \cite{ye2012}. En intégrant $J_{plante}(\omega)$ dans ce modèle, nous pourrons prédire l'impact direct du filtrage spectral sur l'ETR, un excellent indicateur de la productivité photosynthétique \cite{ye2012}.

    \item \textbf{Définition du système quantique ouvert "culture".} Le système sera une unité photosynthétique simplifiée (type FMO/LH2) soumise à deux bains : 1) un bain de photons non-thermique et structuré ($J_{plante}(\omega)$) induisant l'EET, et 2) un bain thermique dissipatif (vibrations) causant la décohérence.
\end{itemize}

\subsection{Aborder le problème de toxicité par l'ingénierie quantique}

L'objectif est d'utiliser les principes quantiques pour guider la conception de matériaux OPV durables.
\begin{itemize}
    \item \textbf{Criblage multi-objectifs par Apprentissage Automatique (ML).} Nous étendrons les approches de criblage à haut débit (HTS) existantes \cite{liu2022, choudhary2019} en intégrant de \textbf{nouveaux descripteurs moléculaires quantifiant la toxicité potentielle et la biodégradabilité}. Le criblage deviendra une optimisation multi-objectifs : maximiser le rendement (PCE) tout en minimisant les scores de toxicité, en s'inspirant de molécules naturelles comme les curcuminoïdes \cite{archet2018}.

    \item \textbf{Lien entre fonction d'état électronique et toxicité/stabilité.} La stabilité photochimique est directement liée à la structure électronique. Une faible bande interdite (gap HOMO-LUMO) et une faible planarité du squelette $\pi$-conjugué, quantifiée par des descripteurs comme \texttt{D\_D/Dtr12} \cite{liu2022}, sont des indicateurs d'instabilité potentielle. Notre approche, centrée sur les matériaux organiques bio-inspirés, permet d'éviter nativement les matériaux inorganiques contenant des métaux lourds toxiques (e.g., CdTe) \cite{archet2018, leb2016}.
\end{itemize}


\section{Méthodologie et cadre de travail}

La force de ce projet réside dans une approche multi-échelle, où les calculs de chimie quantique alimentent des simulations de dynamique quantique de pointe, elles-mêmes guidant une exploration intelligente de l'espace chimique.

\subsection{Paramétrisation du modèle (Input)}

\begin{enumerate}
    \item \textbf{Calculs de structure électronique.} Les énergies des états excités, les moments dipolaires et les couplages seront obtenus par calculs DFT et TD-DFT (ou $\Delta$SCF) avec \texttt{ORCA}. L'inclusion explicite de résidus d'acides aminés sera étudiée, s'inspirant des approches validées sur des centres réactionnels comme celui de \textit{Rhodobacter sphaeroides} \cite{Harush2023}.

    \item \textbf{Caractérisation de l'environnement.} La dynamique moléculaire \textit{ab initio} (AIMD) via \texttt{CP2K} permettra de calculer les fonctions de corrélation du bain thermique, capturant les modes vibrationnels spécifiques du milieu protéique \cite{lee2015, rangel2002}.

    \item \textbf{Décomposition de la fonction de corrélation.} Pour être intégrées au formalisme HOPS, ces fonctions seront décomposées en séries d'exponentielles (Matsubara, Padé) \cite{lambert2023, tao2020}.

    \item \textbf{Modélisation des filtres spectraux.} Pour l'agrivoltaïsme, nous modéliserons la transmission spectrale des panneaux comme un filtre modifiant la densité spectrale incidente : $J_{plante}(\omega) = T(\omega) \cdot J_{soleil}(\omega)$ \cite{Shi2025a}.
\end{enumerate}

\subsection{Simulations de la dynamique quantique ouverte}

\begin{itemize}
    \item \textbf{Implémentation via MesoHOPS.} Le cœur des simulations sera réalisé avec le package Python \texttt{MesoHOPS} \cite{Citty2024}. Son principal avantage est la gestion adaptative de la hiérarchie qui assure une scalabilité en O(1) \cite{varvelo2021}.

    \item \textbf{Modélisation du couplage système-bain.} Nous utiliserons des opérateurs de couplage locaux ($\op{n}$), physiquement bien adaptés pour les matériaux étudiés.

    \item \textbf{Analyse des observables.} Grâce à DadHOPS, nous pourrons calculer les populations, les cohérences et simuler des observables spectroscopiques pour une comparaison directe avec l'expérience \cite{Gera2023, Chen2022a}.

    \item \textbf{Simulations multi-échelles.} Nous combinerons simulations quantiques précises sur petits systèmes et modèles phénoménologiques (basés sur l'ETR \cite{ye2012}) pour les systèmes agrivoltaïques à grande échelle.
\end{itemize}

\subsection{Intégration de méthodes d'intelligence artificielle}

\begin{itemize}
    \item \textbf{Apprentissage Supervisé.} Des modèles (XGBoost, Random Forest) seront entraînés pour prédire les propriétés photovoltaïques et écotoxicologiques à partir de descripteurs moléculaires, en s'appuyant sur des stratégies éprouvées \cite{liu2022, zhang2022_NFA}.

    \item \textbf{Apprentissage Actif (\textit{Active Learning}).} Cette stratégie sera couplée aux simulations AIMD pour optimiser l'échantillonnage des configurations les plus informatives à calculer, réduisant ainsi le coût des calculs \textit{ab initio} \cite{yati2025}.


    \item \textbf{Réseaux de neurones informés par la physique (Physics-Informed Neural Networks, PINNs).} Nous explorerons les PINNs pour garantir des prédictions physiquement cohérentes, assurant que les modèles ML respectent les lois fondamentales de la dynamique quantique \cite{Ullah2024}.
\end{itemize}

\section{Stratégie d'optimisation pour des ressources computationnelles limitées}

Ce projet est conçu pour être mené à bien à l'Université de Yaoundé I.
\begin{itemize}
    \item \textbf{Simulations quantiques maîtrisées.}
        \begin{itemize}
            \item Utilisation de hiérarchies tronquées (\texttt{max\_hierarchy\_depth} entre 3 et 5) et de seuils de convergence raisonnables (\texttt{accuracy=1e-4}).

            \item Focalisation sur des systèmes modèles de petite taille (dimères, trimères) pour valider les protocoles.

            \item Exploitation de la parallélisation MPI native de \texttt{MesoHOPS} sur machines multi-cœurs.
        \end{itemize}

    \item \textbf{Calculs \textit{ab initio} frugaux.}
        \begin{itemize}
            \item Utilisation de bases de fonctions de taille modérée (ex: \texttt{def2-SVP}) pour les calculs DFT/TD-DFT.

            \item Pour l'AIMD, pas de temps plus grands (1-2 fs) et durées de simulation courtes (10-20 ps).
        \end{itemize}
\end{itemize}

%\newpage

\section{Cadre logiciel - 100\% Open-Source}

\begin{table}[htp]
    \centering
    \caption{Cadre logiciel open-source envisagé pour le projet.}
    \label{tab:software}
    \begin{tabular}{@{}lll@{}}
        \toprule
        \textbf{Tâche} & \textbf{Outil principal} & \textbf{Commentaire} \\ \midrule
        Simulations quantiques & \texttt{MesoHOPS}\footnotemark & Framework Python pour adHOPS/DadHOPS. \\
        Calculs DFT/TD-DFT & \texttt{ORCA} & Performant et gratuit pour les académiques. \\
        Dynamique Moléculaire & \texttt{CP2K}/\texttt{QD} & Respectivement pour AIMD. \\
        Apprentissage machine & \texttt{Scikit-learn / XGBoost} & Bibliothèques Python légères et robustes. \\
        Visualisation \& Analyse & \texttt{Matplotlib / Seaborn} & Standards pour la génération de graphiques. \\ \bottomrule
    \end{tabular}
\end{table}
\footnotetext{\url{https://github.com/MesoscienceLab/mesohops.git}}

\newpage

\section{Chronogramme de travail prévisionnel}
\begin{table}[ht]
    \centering
    \caption{Chronogramme de travail prévisionnel.}
    \label{tab:timeline}
    \begin{tabular}{@{}lll@{}}
        \toprule
        \textbf{Période} & \textbf{Tâches principales} & \textbf{Livrables attendus} \\ \midrule
        \textbf{Année 1} & \textbf{Formation, outils \& simulations} & \textbf{Protocole validé} \\
        Mois 1-3 & Apprentissage MesoHOPS, ORCA, CP2K. & \\
        Mois 4-6 & Calculs DFT/TD-DFT sur petites molécules. & Paramètres des Hamiltoniens. \\
        Mois 7-12 & Premières simulations adHOPS. & Communication / Poster. \\ \midrule
        \textbf{Année 2} & \textbf{Simulations avancées, DadHOPS \& IA} & \textbf{Modèles validés} \\
        Mois 13-18 & Application de DadHOPS pour observables. & Comparaison avec l'expérience. \\
        Mois 19-24 & Développement modèles agrivoltaïques. & Modèles prédictifs entraînés. \\ \midrule
        \textbf{Année 3} & \textbf{Analyse fine, simulations et rédaction} & \textbf{Manuscrits \& publications} \\
        Mois 25-30 & Simulations multi-échelles systèmes agrivoltaïques. & \\
        Mois 31-36 & Rédaction du manuscrit et soumission. & \textbf{Thèse, 2-3 articles.} \\ \bottomrule
    \end{tabular}
\end{table}

\section{Impact attendu et perspectives}

Ce projet de thèse ambitionne de générer un impact significatif à plusieurs niveaux :
\begin{itemize}
    \item \textbf{Scientifique.} Décoder finement les phénomènes quantiques non-markoviens qui gouvernent l'efficacité énergétique dans des matériaux complexes \cite{Ablimit2024, Dutta2024}, et fournir des règles de conception moléculaire pour optimiser les systèmes photovoltaïques \cite{Navadel2025}.

    \item \textbf{Technologique.} En approfondissant la compréhension de la photosynthèse, orienter la conception de dispositifs agrivoltaïques optimaux et de photosynthèse artificielle plus efficaces \cite{Tang2023, Yan2023}.

    \item \textbf{Environnemental.} Contribuer au développement de matériaux photovoltaïques non toxiques et biodégradables, réduisant l'impact environnemental de l'énergie solaire \cite{Zeng2025}.

    \item \textbf{Méthodologique.} Promouvoir des méthodes de simulation quantique de pointe, en y intégrant des approches hybrides d'intelligence artificielle \cite{Dan2024, Seneviratne2024}.
\end{itemize}

En alliant rigueur théorique, simulations numériques avancées et une connexion constante avec les défis expérimentaux et sociétaux, cette thèse ouvrira la voie à des découvertes scientifiques et à des innovations technologiques dans le domaine crucial de l'énergie durable \cite{Metzler2023}.

\newpage

\bibliographystyle{unsrt}
\bibliography{Ref_HOPS}

\end{document}

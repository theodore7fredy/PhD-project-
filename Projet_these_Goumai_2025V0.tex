%%%%%%%%%%%%%%%%%%%%%%%%%%%%%%%%%%%%%%%%%%%%%%%%%%%%%%%%%%%%%%%%%%%%%%%%%%%%%%%
% PROJET DE THESE EN DYNAMIQUE QUANTIQUE - adHOPS/DadHOPS
% Rédigé Nana Engo pour Théodore Fredy Goumai
% Date : 16 Septembre 2025
%%%%%%%%%%%%%%%%%%%%%%%%%%%%%%%%%%%%%%%%%%%%%%%%%%%%%%%%%%%%%%%%%%%%%%%%%%%%%%%

\documentclass[12pt, a4paper]{article}

\usepackage[utf8]{inputenc}      
\usepackage[T1]{fontenc}         
\usepackage[french]{babel}       
\usepackage[margin=2.cm]{geometry} 
%
\usepackage{amsmath, amssymb}  
%
\usepackage{graphicx,booktabs,tabularx,hyperref,xcolor,tablefootnote} 
%
\hypersetup{
    colorlinks=true,             % Liens colorés plutôt qu'encadrés
    linkcolor=blue!50!black,     % Couleur des liens internes (sections, figures)
    citecolor=green!50!black,    % Couleur des citations (non utilisé ici, mais bonne pratique)
    urlcolor=blue!80!black,      % Couleur des URLs
    pdftitle={Projet de Thèse - Dynamique Quantique Non-Markovienne},
    pdfauthor={Théodore Fredy Goumai}
}
%
% Ramener la typologie française à la typologie standard (anglo-saxone)
\frenchbsetup{StandardLayout=true}

%
\title{\huge Dynamique quantique non-markovienne du transport et de la séparation de charge : modélisation avancée pour le photovoltaïque organique et la photosynthèse artificielle}
\author{
    Théodore Fredy Goumai (Doctorant) \\
    \\
    \textit{Sous la direction de} \\
    J.-P. Tchapet Njafa, PhD \\
    S. G. Nana Engo, Professeur
}
\date{Octobre 2025}


%%%%%%%%%%%%%%%%%%%%%%%%%%%%%%%%%%%%%%%%%%%%%%%%%%%%%%%%%%%%%%%%%%%%%%%%%%%%%%%
% DÉBUT DU DOCUMENT
%%%%%%%%%%%%%%%%%%%%%%%%%%%%%%%%%%%%%%%%%%%%%%%%%%%%%%%%%%%%%%%%%%%%%%%%%%%%%%%
\begin{document}

\maketitle
\thispagestyle{empty} % Pas de numéro sur la page de titre
\newpage

\tableofcontents % Ajout d'une table des matières
\newpage
\setcounter{page}{1} % Réinitialisation du compteur de pages


\section{Contexte et justification scientifique}

Le développement de matériaux performants pour la conversion photovoltaïque et la photosynthèse artificielle constitue un défi majeur pour la transition énergétique. Au cœur de ces technologies, les processus fondamentaux de transport et de séparation de charge reposent sur une dynamique quantique complexe, intimement couplée à un environnement thermo-vibrationnel \cite{ye2012, mohs2008}. Cet environnement impose souvent des effets de mémoire, dits non-markoviens, que les approches théoriques traditionnelles peinent à capturer fidèlement. Les approximations markoviennes, bien que utiles, s'avèrent souvent insuffisantes pour décrire l'influence des couplages forts et des corrélations temporelles, occultant ainsi le rôle crucial des cohérences quantiques dans l'efficacité des dispositifs \cite{tao2020, dijkstra2010}.

Pour surmonter ces limitations, la méthode de la \textbf{hiérarchie adaptative d'états purs (adHOPS)} a émergé comme une solution numériquement exacte et particulièrement puissante \cite{Suess2014}. Son avantage déterminant réside dans sa capacité à exploiter la localisation dynamique des excitations quantiques et à opérer une réduction adaptative de la base de calcul. Cette approche lui confère une scalabilité quasi-indépendante de la taille du système \cite{varvelo2021}, un atout majeur qui rend possible l'étude de la dynamique quantique ouverte à l'échelle mésoscopique avec un coût computationnel maîtrisé. Les développements récents, notamment la méthode \textbf{DadHOPS (Dyadic adHOPS)} \cite{Gera2023, Chen2022a} et l'implémentation de ces approches dans des frameworks open-source comme \texttt{MesoHOPS} \cite{Citty2024}, démocratisent leur accès et permettent d'envisager des recherches de pointe même dans des environnements aux ressources de calcul modestes \cite{johansson2012}.

\section{Problématique de recherche}

Comment modéliser avec précision et exploiter les interactions fortes et les effets de mémoire non-markoviens dans la dynamique quantique du transport et de la séparation de charges ? Comment l'inclusion explicite de la multiplicité d'états électroniques, notamment les états de transfert de charge (CT) de haute énergie, peut-elle nous guider vers la conception rationnelle de matériaux photovoltaïques et bio-inspirés plus performants ?

\section{Objectifs de la thèse}

Ce projet s'articule autour de deux axes principaux, visant à la fois une compréhension fondamentale et des retombées applicatives.

\subsection{Analyse de l'impact des états de haute énergie dans la séparation de charge}
Cet objectif vise à dépasser les modèles minimaux souvent utilisés.
\begin{itemize}
    \item \textbf{Étude de dyades donneur-accepteur innovantes.} Nous utiliserons adHOPS pour quantifier l'influence des états CT de haute énergie sur la cinétique et l'efficacité de la séparation de charge \cite{lee2015, zhang2022_NFA}.
    
    \item \textbf{Ciblage de matériaux à haut potentiel.} Les simulations se concentreront sur des matériaux bio-inspirés (dérivés de curcuminoïdes, petites molécules fluorées) dont le potentiel de rendement photovoltaïque a été estimé à plus de 20\% \cite{archet2018, firdaus2019, leb2016}.

    \item \textbf{Élucidation des mécanismes fins.} Nous analyserons en détail la dynamique des populations et les voies de recombinaison, en explorant comment les fluctuations vibrationnelles, y compris non-classiques, peuvent assister ou inhiber le processus \cite{oreilly2014, zheng2021, Almazova2024}.
\end{itemize}

\subsection{Optimisation du transfert d'énergie d'excitation dans des systèmes biomimétiques}
Ici, l'objectif est de s'inspirer de l'efficacité remarquable des systèmes naturels.
\begin{itemize}
    \item \textbf{Simulation de complexes photosynthétiques.} Nous modéliserons l'efficacité et la robustesse du transfert d'énergie dans des complexes de référence comme FMO et LH2 \cite{wang2018, Harush2023, Pal2025}, en faisant varier les densités spectrales et les géométries avec adHOPS pour reproduire des conditions réalistes \cite{Zhang2021}.

    \item \textbf{Au-delà des modèles standards.} Nous irons au-delà des densités spectrales de type Drude-Lorentz pour intégrer des fonctions de corrélation plus complexes \cite{rangel2002}, issues de calculs de dynamique moléculaire, afin de mieux représenter la structuration de l'environnement protéique \cite{lambert2023}.

    \item \textbf{Vers des règles de conception.} L'objectif ultime est d'extraire des principes de conception clairs et exploitables pour guider la synthèse de nouveaux matériaux biomimétiques optimisés pour le transport d'énergie \cite{Gauger2024, Peter2024}.
\end{itemize}

\section{Méthodologie et cadre de travail}

La force de ce projet réside dans une approche multi-échelle, où les calculs de chimie quantique alimentent des simulations de dynamique quantique de pointe.

\subsection{Paramétrisation du modèle (Input)}

\begin{enumerate}
    \item \textbf{Calculs de structure électronique.} Les énergies des états excités, les moments dipolaires et les couplages seront obtenus par calculs DFT et TD-DFT (ou $\Delta$SCF) avec \texttt{ORCA}.

    \item \textbf{Caractérisation de l'environnement.} La dynamique moléculaire \textit{ab initio} (AIMD) via \texttt{CP2K} permettra de calculer les fonctions de corrélation du bain thermique \cite{lee2015}.

    \item \textbf{Décomposition de la fonction de corrélation.} Pour être intégrées au formalisme HOPS, ces fonctions seront décomposées en séries d'exponentielles (Matsubara, Padé).
\end{enumerate}

\subsection{Simulations de la dynamique quantique ouverte}

\begin{itemize}
    \item \textbf{Implémentation via MesoHOPS.} Le cœur des simulations sera réalisé avec le package Python \texttt{MesoHOPS} \cite{Citty2024}. Son principal avantage est la gestion adaptative de la hiérarchie.

    \item \textbf{Modélisation du couplage système-bain.} Nous utiliserons des opérateurs de couplage locaux ($|n\rangle\langle n|$), physiquement bien adaptés pour les matériaux étudiés.

    \item \textbf{Analyse des observables.} Grâce à DadHOPS, nous pourrons calculer les populations, les cohérences et simuler des observables spectroscopiques \cite{Gera2023, Chen2022a}.
\end{itemize}

\subsection{Intégration de méthodes d'intelligence artificielle}

\begin{itemize}
    \item \textbf{Apprentissage Supervisé.} Des modèles (XGBoost, Random Forest) seront entraînés pour prédire les propriétés photovoltaïques à partir de descripteurs moléculaires \cite{liu2022}.

    \item \textbf{Apprentissage Actif (\textit{Active Learning}).} Cette stratégie sera couplée aux simulations AIMD pour optimiser l'échantillonnage des configurations les plus informatives à calculer \cite{yati2025}.

    \item \textbf{Réseaux de neurones informés par la physique (PINNs).} Nous explorerons les PINNs pour garantir des prédictions physiquement cohérentes \cite{Ullah2024}.
\end{itemize}

\section{Stratégie d'optimisation pour des ressources computationnelles limitées}

Ce projet est conçu pour être mené à bien à l'Université de Yaoundé I.
\begin{itemize}
    \item \textbf{Simulations quantiques maîtrisées.}
        \begin{itemize}
            \item Utilisation de hiérarchies tronquées (\texttt{max\_hierarchy\_depth} entre 3 et 5) et de seuils de convergence raisonnables (\texttt{accuracy=1e-4}).
            \item Focalisation sur des systèmes modèles de petite taille (dimères, trimères) pour valider les protocoles.
            \item Exploitation de la parallélisation MPI native de \texttt{MesoHOPS} sur machines multi-cœurs.
        \end{itemize}

    \item \textbf{Calculs \textit{ab initio} frugaux.}
        \begin{itemize}
            \item Utilisation de bases de fonctions de taille modérée (ex: \texttt{def2-SVP}) pour les calculs DFT/TD-DFT.
            \item Pour l'AIMD, pas de temps plus grands (1-2 fs) et durées de simulation courtes (10-20 ps).
        \end{itemize}
\end{itemize}

\section{Cadre logiciel - 100\% Open-Source}
\begin{table}[htp]
    \centering
    \caption{Cadre logiciel open-source envisagé pour le projet.}
    \label{tab:software}
    \begin{tabular}{@{}lll@{}}
        \toprule
        \textbf{Tâche} & \textbf{Outil principal} & \textbf{Commentaire} \\ \midrule
        Simulations quantiques & \texttt{MesoHOPS}\footnotemark & Framework Python pour adHOPS/DadHOPS. \\
        Calculs DFT/TD-DFT & \texttt{ORCA} & Performant et gratuit pour les académiques. \\
        Dynamique Moléculaire & \texttt{CP2K} & Respectivement pour AIMD. \\
        Apprentissage machine & \texttt{Scikit-learn / XGBoost} & Bibliothèques Python légères et robustes. \\
        Visualisation \& Analyse & \texttt{Matplotlib / Seaborn} & Standards pour la génération de graphiques. \\ \bottomrule
    \end{tabular}
\end{table}
\footnotetext{\url{https://github.com/MesoscienceLab/mesohops.git}}

\newpage

\section{Chronogramme de travail prévisionnel}
\begin{table}[ht]
    \centering
    \caption{Chronogramme de travail prévisionnel.}
    \label{tab:timeline}
    \begin{tabular}{@{}lll@{}}
        \toprule
        \textbf{Période} & \textbf{Tâches principales} & \textbf{Livrables attendus} \\ \midrule
        \textbf{Année 1} & \textbf{Formation, outils \& simulations} & \textbf{Protocole validé} \\
        Mois 1-3 & Apprentissage MesoHOPS, ORCA, CP2K. & \\
        Mois 4-6 & Calculs DFT/TD-DFT sur petites molécules. & Paramètres des Hamiltoniens. \\
        Mois 7-12 & Premières simulations adHOPS. & Communication / Poster. \\ \midrule
        \textbf{Année 2} & \textbf{Simulations avancées, DadHOPS \& IA} & \textbf{Modèles validés} \\
        Mois 13-18 & Application de DadHOPS pour observables. & Comparaison avec l'expérience. \\
        Mois 19-24 & Développement de l'Active Learning. & Modèles prédictifs entraînés. \\ \midrule
        \textbf{Année 3} & \textbf{Analyse fine, simulations et rédaction} & \textbf{Manuscrits \& publications} \\
        Mois 25-30 & Simulations sur systèmes mésoscopiques. & \\
        Mois 31-36 & Rédaction du manuscrit et soumission. & \textbf{Thèse, 2-3 articles.} \\ \bottomrule
    \end{tabular}
\end{table}

\section{Impact attendu et perspectives}

Ce projet de thèse ambitionne de générer un impact significatif à plusieurs niveaux :
\begin{itemize}
    \item \textbf{Scientifique.} Décoder finement les phénomènes quantiques non-markoviens qui gouvernent l'efficacité énergétique dans des matériaux complexes \cite{Ablimit2024, Dutta2024}, et fournir des règles de conception moléculaire pour optimiser les systèmes photovoltaïques \cite{Navadel2025}.

    \item \textbf{Technologique.} En approfondissant la compréhension de la photosynthèse, orienter la conception de dispositifs de photosynthèse artificielle plus efficaces \cite{Tang2023, Yan2023}.

    \item \textbf{Méthodologique.} Promouvoir des méthodes de simulation quantique de pointe, en y intégrant des approches hybrides d'intelligence artificielle \cite{Dan2024, Seneviratne2024}.
\end{itemize}

En alliant rigueur théorique, simulations numériques avancées et une connexion constante avec les défis expérimentaux, cette thèse ouvrira la voie à des découvertes scientifiques et à des innovations technologiques dans le domaine crucial de l'énergie durable \cite{Metzler2023}.

\newpage

\bibliographystyle{unsrt}
\bibliography{Ref_HOPS}

\end{document}

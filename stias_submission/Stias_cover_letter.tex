\documentclass[11pt,a4paper]{letter}
\usepackage[utf8]{inputenc}
\usepackage[T1]{fontenc}
\usepackage{lmodern,graphicx}
\usepackage[english]{babel}
\usepackage[margin=2cm]{geometry}
\usepackage{hyperref}

\hypersetup{
    colorlinks=true,
    linkcolor=blue!70!black,
    urlcolor=blue!80!black
}

% Custom command for email to ensure consistent formatting
\newcommand{\email}[1]{\texttt{#1}}

\begin{document}

\begin{letter}{The Selection Committee\\ Stellenbosch Institute for Advanced Study (STIAS)\\ Wallenberg Research Centre\\ 10 Marais Road, Mostertsdrift\\ Stellenbosch, South Africa}

\opening{Dear Members of the Selection Committee,}

On behalf of the team led by Dr. Theodore GOUMAI VEDEKOI, I am pleased to submit this application for a STIAS Team Fellowship for the First Semester 2027. We propose an intensive, three-month residency from 1 March to 31 May 2027, to launch a new scientific paradigm we call "\textbf{Quantum Agrivoltaics}".

Our team brings together a synergistic blend of expertise: Mr. Theodore GOUMAI VEDEKOI, our exceptional PhD candidate and the engine of this project; Dr. J.-P. Tchapet Njafa, our lead theorist in materials science and systems engineering; and myself, guiding the quantum dynamics, materials science and systems engineering. We were drawn to STIAS following a strong recommendation from Dr. Philippe Djorwe—my first doctoral student, now an Iso-Lomso Fellow—who described it as the ideal intellectual crucible for the kind of deep, interdisciplinary work our project demands.

Our mission is to unravel the quantum rules that govern the delicate symbiosis between solar energy generation and agriculture. The STIAS residency would provide the focused, collaborative environment essential for us to achieve our most critical goals: finalizing our open-source software platform, the \textit{AgroQuantPV Suite}; conducting hands-on experiments with our unique quantum sensors; and convening a multi-disciplinary workshop with local experts to lay the groundwork for transformative field trials.

We are confident that this residency will culminate in a validated software release, a high-impact public seminar, and a concrete roadmap for translating our fundamental research into real-world solutions. More importantly, it will allow us to forge the connections and refine the ideas that will drive this new field forward.

We stand ready to provide any additional documentation and would be honored to discuss our vision with you further. Thank you for your time and consideration. We eagerly await the opportunity to contribute our passion and expertise to the vibrant scholarly community at STIAS.

Sincerely,

\includegraphics[width=.25\textwidth]{/home/taamangtchu/Images/NESG_Signature.png}

% \vspace{.5em}
\noindent\textbf{Serge Guy Nana Engo} \\
Professor \& Principal Investigator \\
Department of Physics, Faculty of Science \\
University of Yaoundé I, Cameroon \\
Email: \email{serge.nana-engo@facsciences-uy1.cm} \\
Phone: +237 670 511 632

\end{letter}
\end{document}